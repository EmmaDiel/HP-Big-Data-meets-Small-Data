% Options for packages loaded elsewhere
\PassOptionsToPackage{unicode}{hyperref}
\PassOptionsToPackage{hyphens}{url}
%
\documentclass[
]{article}
\usepackage{amsmath,amssymb}
\usepackage{iftex}
\ifPDFTeX
  \usepackage[T1]{fontenc}
  \usepackage[utf8]{inputenc}
  \usepackage{textcomp} % provide euro and other symbols
\else % if luatex or xetex
  \usepackage{unicode-math} % this also loads fontspec
  \defaultfontfeatures{Scale=MatchLowercase}
  \defaultfontfeatures[\rmfamily]{Ligatures=TeX,Scale=1}
\fi
\usepackage{lmodern}
\ifPDFTeX\else
  % xetex/luatex font selection
\fi
% Use upquote if available, for straight quotes in verbatim environments
\IfFileExists{upquote.sty}{\usepackage{upquote}}{}
\IfFileExists{microtype.sty}{% use microtype if available
  \usepackage[]{microtype}
  \UseMicrotypeSet[protrusion]{basicmath} % disable protrusion for tt fonts
}{}
\makeatletter
\@ifundefined{KOMAClassName}{% if non-KOMA class
  \IfFileExists{parskip.sty}{%
    \usepackage{parskip}
  }{% else
    \setlength{\parindent}{0pt}
    \setlength{\parskip}{6pt plus 2pt minus 1pt}}
}{% if KOMA class
  \KOMAoptions{parskip=half}}
\makeatother
\usepackage{xcolor}
\usepackage[margin=1in]{geometry}
\usepackage{color}
\usepackage{fancyvrb}
\newcommand{\VerbBar}{|}
\newcommand{\VERB}{\Verb[commandchars=\\\{\}]}
\DefineVerbatimEnvironment{Highlighting}{Verbatim}{commandchars=\\\{\}}
% Add ',fontsize=\small' for more characters per line
\usepackage{framed}
\definecolor{shadecolor}{RGB}{248,248,248}
\newenvironment{Shaded}{\begin{snugshade}}{\end{snugshade}}
\newcommand{\AlertTok}[1]{\textcolor[rgb]{0.94,0.16,0.16}{#1}}
\newcommand{\AnnotationTok}[1]{\textcolor[rgb]{0.56,0.35,0.01}{\textbf{\textit{#1}}}}
\newcommand{\AttributeTok}[1]{\textcolor[rgb]{0.77,0.63,0.00}{#1}}
\newcommand{\BaseNTok}[1]{\textcolor[rgb]{0.00,0.00,0.81}{#1}}
\newcommand{\BuiltInTok}[1]{#1}
\newcommand{\CharTok}[1]{\textcolor[rgb]{0.31,0.60,0.02}{#1}}
\newcommand{\CommentTok}[1]{\textcolor[rgb]{0.56,0.35,0.01}{\textit{#1}}}
\newcommand{\CommentVarTok}[1]{\textcolor[rgb]{0.56,0.35,0.01}{\textbf{\textit{#1}}}}
\newcommand{\ConstantTok}[1]{\textcolor[rgb]{0.00,0.00,0.00}{#1}}
\newcommand{\ControlFlowTok}[1]{\textcolor[rgb]{0.13,0.29,0.53}{\textbf{#1}}}
\newcommand{\DataTypeTok}[1]{\textcolor[rgb]{0.13,0.29,0.53}{#1}}
\newcommand{\DecValTok}[1]{\textcolor[rgb]{0.00,0.00,0.81}{#1}}
\newcommand{\DocumentationTok}[1]{\textcolor[rgb]{0.56,0.35,0.01}{\textbf{\textit{#1}}}}
\newcommand{\ErrorTok}[1]{\textcolor[rgb]{0.64,0.00,0.00}{\textbf{#1}}}
\newcommand{\ExtensionTok}[1]{#1}
\newcommand{\FloatTok}[1]{\textcolor[rgb]{0.00,0.00,0.81}{#1}}
\newcommand{\FunctionTok}[1]{\textcolor[rgb]{0.00,0.00,0.00}{#1}}
\newcommand{\ImportTok}[1]{#1}
\newcommand{\InformationTok}[1]{\textcolor[rgb]{0.56,0.35,0.01}{\textbf{\textit{#1}}}}
\newcommand{\KeywordTok}[1]{\textcolor[rgb]{0.13,0.29,0.53}{\textbf{#1}}}
\newcommand{\NormalTok}[1]{#1}
\newcommand{\OperatorTok}[1]{\textcolor[rgb]{0.81,0.36,0.00}{\textbf{#1}}}
\newcommand{\OtherTok}[1]{\textcolor[rgb]{0.56,0.35,0.01}{#1}}
\newcommand{\PreprocessorTok}[1]{\textcolor[rgb]{0.56,0.35,0.01}{\textit{#1}}}
\newcommand{\RegionMarkerTok}[1]{#1}
\newcommand{\SpecialCharTok}[1]{\textcolor[rgb]{0.00,0.00,0.00}{#1}}
\newcommand{\SpecialStringTok}[1]{\textcolor[rgb]{0.31,0.60,0.02}{#1}}
\newcommand{\StringTok}[1]{\textcolor[rgb]{0.31,0.60,0.02}{#1}}
\newcommand{\VariableTok}[1]{\textcolor[rgb]{0.00,0.00,0.00}{#1}}
\newcommand{\VerbatimStringTok}[1]{\textcolor[rgb]{0.31,0.60,0.02}{#1}}
\newcommand{\WarningTok}[1]{\textcolor[rgb]{0.56,0.35,0.01}{\textbf{\textit{#1}}}}
\usepackage{graphicx}
\makeatletter
\def\maxwidth{\ifdim\Gin@nat@width>\linewidth\linewidth\else\Gin@nat@width\fi}
\def\maxheight{\ifdim\Gin@nat@height>\textheight\textheight\else\Gin@nat@height\fi}
\makeatother
% Scale images if necessary, so that they will not overflow the page
% margins by default, and it is still possible to overwrite the defaults
% using explicit options in \includegraphics[width, height, ...]{}
\setkeys{Gin}{width=\maxwidth,height=\maxheight,keepaspectratio}
% Set default figure placement to htbp
\makeatletter
\def\fps@figure{htbp}
\makeatother
\setlength{\emergencystretch}{3em} % prevent overfull lines
\providecommand{\tightlist}{%
  \setlength{\itemsep}{0pt}\setlength{\parskip}{0pt}}
\setcounter{secnumdepth}{-\maxdimen} % remove section numbering
\ifLuaTeX
  \usepackage{selnolig}  % disable illegal ligatures
\fi
\IfFileExists{bookmark.sty}{\usepackage{bookmark}}{\usepackage{hyperref}}
\IfFileExists{xurl.sty}{\usepackage{xurl}}{} % add URL line breaks if available
\urlstyle{same}
\hypersetup{
  pdftitle={Basics of Data Visualization},
  pdfauthor={Kasper Welbers, Wouter van Atteveldt \& Philipp Masur},
  hidelinks,
  pdfcreator={LaTeX via pandoc}}

\title{Basics of Data Visualization}
\author{Kasper Welbers, Wouter van Atteveldt \& Philipp Masur}
\date{2021-10}

\begin{document}
\maketitle

{
\setcounter{tocdepth}{2}
\tableofcontents
}
This tutorial teaches the basics of data visualization using the
\texttt{ggplot2} package (included in the \texttt{tidyverse}). For more
information, see
\href{http://r4ds.had.co.nz/data-visualisation.html}{R4DS Chapter 3:
Data Visualization} and
\href{http://r4ds.had.co.nz/exploratory-data-analysis.html}{R4DS Chapter
7: Exploratory Data Analysis}.

For \emph{many} cool visualization examples using \texttt{ggplot2} (with
R code included!) see the
\href{https://www.r-graph-gallery.com/portfolio/ggplot2-package/}{R
Graph Gallery}. For inspiration (but unfortunately no R code), there is
also a
\href{https://fivethirtyeight.com/features/the-52-best-and-weirdest-charts-we-made-in-2016/}{538
blog post on data visualization from 2016}. Finally, see the article on
`\href{http://vita.had.co.nz/papers/layered-grammar.html}{the grammar of
graphics}' published by Hadley Wickham for more insight into the ideas
behind ggplot.

\hypertarget{a-basic-ggplot-plot}{%
\section{A basic ggplot plot}\label{a-basic-ggplot-plot}}

\hypertarget{loading-packages-and-data}{%
\subsection{Loading packages and data}\label{loading-packages-and-data}}

Suppose that we want to see the relation between college education and
household income, both included in the \texttt{county\ facts} subset
published by
\href{https://github.com/houstondatavis/data-jam-august-2016}{Houston
Data Visualisation github page}.

(If you want to practice downloading a data set into a folder and
loading it from there: you can find the data set on Canvas as well. Bear
in mind to set the working directory correctly).

\begin{Shaded}
\begin{Highlighting}[]
\CommentTok{\# Load package collection}
\FunctionTok{library}\NormalTok{(tidyverse)}

\CommentTok{\# Download data}
\NormalTok{csv\_folder\_url }\OtherTok{\textless{}{-}} \StringTok{"https://raw.githubusercontent.com/houstondatavis/data{-}jam{-}august{-}2016/master/csv"}  \CommentTok{\# URL to folder }
\NormalTok{facts }\OtherTok{\textless{}{-}} \FunctionTok{read\_csv}\NormalTok{(}\FunctionTok{paste}\NormalTok{(csv\_folder\_url, }\StringTok{"county\_facts.csv"}\NormalTok{, }\AttributeTok{sep =} \StringTok{"/"}\NormalTok{)) }\CommentTok{\# pasting folder path and file name together}
\NormalTok{facts}
\end{Highlighting}
\end{Shaded}

Since this data set contains a large amount of columns, we keep only a
subset of columns for now:

\begin{Shaded}
\begin{Highlighting}[]
\CommentTok{\# Selecting columns and filtering }
\NormalTok{facts\_state }\OtherTok{\textless{}{-}}\NormalTok{ facts }\SpecialCharTok{\%\textgreater{}\%} 
  \FunctionTok{select}\NormalTok{(fips, area\_name, state\_abbreviation, }
         \AttributeTok{population =}\NormalTok{ Pop\_2014\_count, }
         \AttributeTok{pop\_change =}\NormalTok{ Pop\_change\_pct,}
         \AttributeTok{over65 =}\NormalTok{ Age\_over\_65\_pct, }
         \AttributeTok{female =}\NormalTok{ Sex\_female\_pct,}
         \AttributeTok{white =}\NormalTok{ Race\_white\_pct,}
         \AttributeTok{college =}\NormalTok{ Pop\_college\_grad\_pct, }
         \AttributeTok{income =}\NormalTok{ Income\_per\_capita) }\SpecialCharTok{\%\textgreater{}\%}
  \FunctionTok{filter}\NormalTok{(}\FunctionTok{is.na}\NormalTok{(state\_abbreviation) }\SpecialCharTok{\&}\NormalTok{ fips }\SpecialCharTok{!=} \DecValTok{0}\NormalTok{) }\SpecialCharTok{\%\textgreater{}\%} 
  \FunctionTok{select}\NormalTok{(}\SpecialCharTok{{-}}\NormalTok{state\_abbreviation)}

\CommentTok{\# Check results}
\NormalTok{facts\_state}
\end{Highlighting}
\end{Shaded}

\hypertarget{building-a-layered-visualization}{%
\subsection{Building a layered
visualization}\label{building-a-layered-visualization}}

Now, let's make a simple \emph{scatter plot} with percentage
college-educated on the x-axis and median income on the y-axis. First,
we can used the function \texttt{ggplot} to create an empty canvas tied
to the dataset \texttt{facts\_state} and tell the function which
variables to use:

\begin{Shaded}
\begin{Highlighting}[]
\FunctionTok{ggplot}\NormalTok{(}\AttributeTok{data =}\NormalTok{ facts\_state,        }\DocumentationTok{\#\# which data set?}
       \FunctionTok{aes}\NormalTok{(}\AttributeTok{x=}\NormalTok{college, }\AttributeTok{y=}\NormalTok{income))  }\DocumentationTok{\#\# which variables as aesthetics?}
\end{Highlighting}
\end{Shaded}

Next, we need to tell ggplot what to plot. In this case, we want to
produce a scatterplot. The function \texttt{geom\_point} adds a layer of
information to the canvas. In the language of ggplot, each layer has a
\emph{geometrical representation}, in this case ``points''. In this
case, the ``x'' and ``y'' are mapped to the college and income columns.

\begin{Shaded}
\begin{Highlighting}[]
\FunctionTok{ggplot}\NormalTok{(}\AttributeTok{data =}\NormalTok{ facts\_state,}
       \AttributeTok{mapping =} \FunctionTok{aes}\NormalTok{(}\AttributeTok{x =}\NormalTok{ college, }\AttributeTok{y =}\NormalTok{ income)) }\SpecialCharTok{+} 
  \FunctionTok{geom\_point}\NormalTok{()   }\DocumentationTok{\#\# adding the geometrical representation}
\end{Highlighting}
\end{Shaded}

The result is a plot where each point here represents a state, and we
see a clear correlation between education level and income. There is one
clear outlier on the top-right. Can you guess which state that is?

So called \emph{aesthetic mappings}, which map the visual elements of
the geometry to columns of the data, can also be included as argument in
the \texttt{geom} itself and not in the \texttt{ggplot()?} command. This
can be handy when several \texttt{geoms} are plotted and different
aesthetics are used. For example, we can add more \texttt{geoms} to the
plot (e.g., a regression line). If we provided the aesthetics within the
\texttt{ggplot}-function, these are passed automatically to the
following \texttt{geoms}.

\begin{Shaded}
\begin{Highlighting}[]
\CommentTok{\# Linear regression line}
\FunctionTok{ggplot}\NormalTok{(}\AttributeTok{data =}\NormalTok{ facts\_state, }
       \AttributeTok{mapping =} \FunctionTok{aes}\NormalTok{(}\AttributeTok{x =}\NormalTok{ college, }\AttributeTok{y =}\NormalTok{ income)) }\SpecialCharTok{+} 
  \FunctionTok{geom\_point}\NormalTok{() }\SpecialCharTok{+}
  \FunctionTok{geom\_smooth}\NormalTok{(}\AttributeTok{method =} \StringTok{"lm"}\NormalTok{)}
\end{Highlighting}
\end{Shaded}

\hypertarget{important-note-on-ggplot-command-syntax}{%
\subsection{Important note on ggplot command
syntax}\label{important-note-on-ggplot-command-syntax}}

For the plot to work, R needs to execute the whole ggplot call and all
layers as a single statement. Practically, that means that if you
combine a plot over multiple lines, the plus sign needs to be at the end
of the line, so R knows more is coming.

So, the following is good:

\begin{Shaded}
\begin{Highlighting}[]
\FunctionTok{ggplot}\NormalTok{(}\AttributeTok{data =}\NormalTok{ facts\_state,}
       \AttributeTok{mapping =} \FunctionTok{aes}\NormalTok{(}\AttributeTok{x =}\NormalTok{ college, }\AttributeTok{y =}\NormalTok{ income)) }\SpecialCharTok{+} 
  \FunctionTok{geom\_point}\NormalTok{()}
\end{Highlighting}
\end{Shaded}

But this is not:

\begin{Shaded}
\begin{Highlighting}[]
\FunctionTok{ggplot}\NormalTok{(}\AttributeTok{data =}\NormalTok{ facts\_state,}
       \AttributeTok{mapping =} \FunctionTok{aes}\NormalTok{(}\AttributeTok{x =}\NormalTok{ college, }\AttributeTok{y =}\NormalTok{ income)) }
  \SpecialCharTok{+} \FunctionTok{geom\_point}\NormalTok{()}
\end{Highlighting}
\end{Shaded}

We can also move the mappings to the \texttt{geom}. This can be useful
when we want to plot different \texttt{geoms} based on different
variables.

\begin{Shaded}
\begin{Highlighting}[]
\CommentTok{\# same as above}
\FunctionTok{ggplot}\NormalTok{(}\AttributeTok{data =}\NormalTok{ facts\_state) }\SpecialCharTok{+} 
  \FunctionTok{geom\_point}\NormalTok{(}\AttributeTok{mapping =} \FunctionTok{aes}\NormalTok{(}\AttributeTok{x =}\NormalTok{ college, }\AttributeTok{y =}\NormalTok{ income))}
\end{Highlighting}
\end{Shaded}

Also note that the data and mapping arguments are always the first
arguments the functions expect, so you can also call them implicitly:

\begin{Shaded}
\begin{Highlighting}[]
\FunctionTok{ggplot}\NormalTok{(facts\_state) }\SpecialCharTok{+} 
  \FunctionTok{geom\_point}\NormalTok{(}\FunctionTok{aes}\NormalTok{(}\AttributeTok{x =}\NormalTok{ college, }\AttributeTok{y =}\NormalTok{ income))}
\end{Highlighting}
\end{Shaded}

\hypertarget{other-aesthetics}{%
\subsection{Other aesthetics}\label{other-aesthetics}}

To find out which visual elements can be used in a layer, use
e.g.~\texttt{?geom\_point}. According to the help file, we can (among
others) set the colour, alpha (transparency), and size of points. Let's
first set the size of points to the population of each state, creating a
bubble plot:

\begin{Shaded}
\begin{Highlighting}[]
\FunctionTok{ggplot}\NormalTok{(}\AttributeTok{data =}\NormalTok{ facts\_state) }\SpecialCharTok{+} 
  \FunctionTok{geom\_point}\NormalTok{(}\FunctionTok{aes}\NormalTok{(}\AttributeTok{x =}\NormalTok{ college, }\AttributeTok{y =}\NormalTok{ income, }\AttributeTok{size =}\NormalTok{ population))}
\end{Highlighting}
\end{Shaded}

Since it is difficult to see overlapping points, let's make all points
somewhat transparent. Note: Since we want to set the alpha of all points
to a single value, this is not a mapping (as it is not mapped to a
column from the data frame), but a constant. These are set outside the
mapping argument:

\begin{Shaded}
\begin{Highlighting}[]
\FunctionTok{ggplot}\NormalTok{(}\AttributeTok{data =}\NormalTok{ facts\_state) }\SpecialCharTok{+} 
  \FunctionTok{geom\_point}\NormalTok{(}\FunctionTok{aes}\NormalTok{(}\AttributeTok{x =}\NormalTok{ college, }\AttributeTok{y =}\NormalTok{ income, }\AttributeTok{size =}\NormalTok{ population), }
             \AttributeTok{alpha =}\NormalTok{ .}\DecValTok{5}\NormalTok{, }
             \AttributeTok{colour =} \StringTok{"red"}\NormalTok{)}
\end{Highlighting}
\end{Shaded}

Instead of setting colour to a constant value, we can also let it vary
with the data. For example, we can colour the states by percentage of
population above 65:

\begin{Shaded}
\begin{Highlighting}[]
\FunctionTok{ggplot}\NormalTok{(}\AttributeTok{data =}\NormalTok{ facts\_state) }\SpecialCharTok{+} 
  \FunctionTok{geom\_point}\NormalTok{(}\FunctionTok{aes}\NormalTok{(}\AttributeTok{x =}\NormalTok{ college, }\AttributeTok{y =}\NormalTok{ income, }\AttributeTok{size =}\NormalTok{ population, }\AttributeTok{colour =}\NormalTok{ over65), }
             \AttributeTok{alpha =}\NormalTok{ .}\DecValTok{9}\NormalTok{)}
\end{Highlighting}
\end{Shaded}

Finally, you can map to a categorical value as well. Let's categorize
states into whether population is growing (at least 1\%) or stable or
declining. We use the \texttt{if\_else(condition,\ iftrue,\ iffalse)}
function, which assigns the \texttt{iftrue} value if the condition is
true, and \texttt{iffalse} otherwise:

\begin{Shaded}
\begin{Highlighting}[]
\CommentTok{\# Creating a new variable}
\NormalTok{facts\_state }\OtherTok{\textless{}{-}}\NormalTok{ facts\_state }\SpecialCharTok{\%\textgreater{}\%} 
  \FunctionTok{mutate}\NormalTok{(}\AttributeTok{growth =} \FunctionTok{ifelse}\NormalTok{(pop\_change }\SpecialCharTok{\textgreater{}} \DecValTok{1}\NormalTok{, }\StringTok{"Growing"}\NormalTok{, }\StringTok{"Stable"}\NormalTok{))}

\CommentTok{\# Plotting a categorical variable}
\FunctionTok{ggplot}\NormalTok{(}\AttributeTok{data=}\NormalTok{facts\_state) }\SpecialCharTok{+} 
  \FunctionTok{geom\_point}\NormalTok{(}\FunctionTok{aes}\NormalTok{(}\AttributeTok{x =}\NormalTok{ college, }\AttributeTok{y =}\NormalTok{ income, }\AttributeTok{size =}\NormalTok{ population, }\AttributeTok{colour =}\NormalTok{ growth), }
             \AttributeTok{alpha=}\NormalTok{.}\DecValTok{9}\NormalTok{)}
\end{Highlighting}
\end{Shaded}

As you can see in these examples, ggplot tries to be smart about the
mapping you ask. It automatically sets the x and y ranges to the values
in your data. It mapped the size such that there are small and large
points, but not e.g.~a point so large that it would dominate the graph.
For the colour, for interval variables it created a colour scale, while
for a categorical variable it automatically assigned a colour to each
group.

Of course, each of those choices can be customized, and sometimes it
makes a lot of sense to do so. For example, you might wish to use red
for republicans and blue for democrats, if your audience is used to
those colors; or you may wish to use grayscale for an old-fashioned
paper publication. We'll explore more options in a later tutorial, but
for now let's be happy that ggplot does a lot of work for us!

\textbf{Exercise 1:} Of course there are more types of plots. Try to
plot a so-called ``histogram'' of the variable \texttt{income}.
Histograms are really helpful to understand the distribution of a
variable. Tip: Check out the help page by calling
\texttt{?geom\_histogram}. The first example will help you build the
right plot. In a second step, simply exchange \texttt{geom\_histogram()}
with \texttt{geom\_density()}. What do you see now?

\begin{Shaded}
\begin{Highlighting}[]
\CommentTok{\# Solution here}
\end{Highlighting}
\end{Shaded}

\hypertarget{bar-plots}{%
\section{Bar plots}\label{bar-plots}}

Another frequently used plot is the bar plot. By default, R bar plots
assume that you want to plot a histogram, e.g.~the number of occurences
of each group. As a very simple example, the following plots the number
of states that are growing or stable in population:

\begin{Shaded}
\begin{Highlighting}[]
\FunctionTok{ggplot}\NormalTok{(}\AttributeTok{data =}\NormalTok{ facts\_state) }\SpecialCharTok{+} 
  \FunctionTok{geom\_bar}\NormalTok{(}\FunctionTok{aes}\NormalTok{(}\AttributeTok{x =}\NormalTok{ growth))}
\end{Highlighting}
\end{Shaded}

For a more interesting plot, let's plot the votes per Republican
candidate in the New Hampshire primary. First, we need to download the
per-county data, summarize it per state, and filter to only get the NH
results for the Republican party: (see the previous tutorials on
\href{R-tidy-5-transformation.md}{Data Transformations} and
\href{R-tidy-13a-joining.md}{Joining data} for more information if
needed)

\begin{Shaded}
\begin{Highlighting}[]
\CommentTok{\# Getting new data}
\NormalTok{results\_state }\OtherTok{\textless{}{-}} \FunctionTok{read\_csv}\NormalTok{(}\FunctionTok{paste}\NormalTok{(csv\_folder\_url, }\StringTok{"primary\_results.csv"}\NormalTok{, }\AttributeTok{sep =} \StringTok{"/"}\NormalTok{)) }\SpecialCharTok{\%\textgreater{}\%} 
  \FunctionTok{group\_by}\NormalTok{(state, party, candidate) }\SpecialCharTok{\%\textgreater{}\%} 
  \FunctionTok{summarize}\NormalTok{(}\AttributeTok{votes=}\FunctionTok{sum}\NormalTok{(votes))}

\CommentTok{\# Subset of New Hampshire and republican candidates}
\NormalTok{nh\_gop }\OtherTok{\textless{}{-}}\NormalTok{ results\_state }\SpecialCharTok{\%\textgreater{}\%} 
  \FunctionTok{filter}\NormalTok{(state }\SpecialCharTok{==} \StringTok{"New Hampshire"} \SpecialCharTok{\&}\NormalTok{ party }\SpecialCharTok{==} \StringTok{"Republican"}\NormalTok{)}
\NormalTok{nh\_gop}
\end{Highlighting}
\end{Shaded}

Now, let's make a bar plot with votes (y) per candidate (x). Since we
don't want ggplot to summarize it for us (we already did that
ourselves), we set \texttt{stat="identity"} to set the grouping
statistic to the identity function, i.e.~just use each point as it is.

\begin{Shaded}
\begin{Highlighting}[]
\CommentTok{\# We can also store parts of a plot in an object}
\NormalTok{plot1 }\OtherTok{\textless{}{-}} \FunctionTok{ggplot}\NormalTok{(nh\_gop) }\SpecialCharTok{+} 
  \FunctionTok{geom\_bar}\NormalTok{(}\FunctionTok{aes}\NormalTok{(}\AttributeTok{x=}\NormalTok{candidate, }\AttributeTok{y=}\NormalTok{votes), }
           \AttributeTok{stat=}\StringTok{\textquotesingle{}identity\textquotesingle{}}\NormalTok{)}
\NormalTok{plot1}
\end{Highlighting}
\end{Shaded}

\hypertarget{setting-graph-options}{%
\subsection{Setting graph options}\label{setting-graph-options}}

Some options, like labels, legends, and the coordinate system are
graph-wide rather than per layer. You add these options to the graph by
adding extra functions to the call. For example, we can use
coord\_flip() to swap the x and y axes:

\begin{Shaded}
\begin{Highlighting}[]
\NormalTok{plot1 }\SpecialCharTok{+} 
  \FunctionTok{coord\_flip}\NormalTok{()}
\end{Highlighting}
\end{Shaded}

You can also reorder categories with the \texttt{reorder} function, for
example to sort by number of votes. Also, let's add some colour (just
because we can!):

\begin{Shaded}
\begin{Highlighting}[]
\FunctionTok{ggplot}\NormalTok{(nh\_gop) }\SpecialCharTok{+} 
  \FunctionTok{geom\_bar}\NormalTok{(}\FunctionTok{aes}\NormalTok{(}\AttributeTok{x =} \FunctionTok{reorder}\NormalTok{(candidate, votes), }\AttributeTok{y =}\NormalTok{ votes, }\AttributeTok{fill =}\NormalTok{ candidate), }
           \AttributeTok{stat =} \StringTok{\textquotesingle{}identity\textquotesingle{}}\NormalTok{) }\SpecialCharTok{+} 
  \FunctionTok{coord\_flip}\NormalTok{()}
\end{Highlighting}
\end{Shaded}

This is getting somewhere, but the x-axis label (y-axis after rotation)
is not very pretty and we don't need guides for the fill mapping. This
can be remedied by more graph-level options. Also, we can use a
\texttt{theme} to alter the appearance of the graph, for example using
the minimal theme:

\begin{Shaded}
\begin{Highlighting}[]
\FunctionTok{ggplot}\NormalTok{(nh\_gop) }\SpecialCharTok{+} 
  \FunctionTok{geom\_bar}\NormalTok{(}\FunctionTok{aes}\NormalTok{(}\AttributeTok{x =} \FunctionTok{reorder}\NormalTok{(candidate, votes), }\AttributeTok{y =}\NormalTok{ votes, }\AttributeTok{fill =}\NormalTok{ candidate), }
           \AttributeTok{stat =} \StringTok{\textquotesingle{}identity\textquotesingle{}}\NormalTok{) }\SpecialCharTok{+} 
  \FunctionTok{coord\_flip}\NormalTok{() }\SpecialCharTok{+}
  \FunctionTok{xlab}\NormalTok{(}\StringTok{"Candidate"}\NormalTok{) }\SpecialCharTok{+} 
  \FunctionTok{theme\_minimal}\NormalTok{() }\SpecialCharTok{+}
  \FunctionTok{theme}\NormalTok{(}\AttributeTok{legend.position =} \StringTok{"none"}\NormalTok{)}
\end{Highlighting}
\end{Shaded}

\hypertarget{grouped-bar-plots}{%
\subsection{Grouped bar plots}\label{grouped-bar-plots}}

We can also add groups to bar plots. For example, we can set the x
category to state (taking only NH and IA to keep the plot readable), and
then group by candidate:

\begin{Shaded}
\begin{Highlighting}[]
\NormalTok{gop2 }\OtherTok{\textless{}{-}}\NormalTok{ results\_state }\SpecialCharTok{\%\textgreater{}\%} 
  \FunctionTok{filter}\NormalTok{(party }\SpecialCharTok{==} \StringTok{"Republican"} \SpecialCharTok{\&}\NormalTok{ (state }\SpecialCharTok{==} \StringTok{"New Hampshire"} \SpecialCharTok{|}\NormalTok{ state }\SpecialCharTok{==} \StringTok{"Iowa"}\NormalTok{)) }

\FunctionTok{ggplot}\NormalTok{(gop2) }\SpecialCharTok{+} 
  \FunctionTok{geom\_bar}\NormalTok{(}\FunctionTok{aes}\NormalTok{(}\AttributeTok{x =}\NormalTok{ state, }\AttributeTok{y =}\NormalTok{ votes, }\AttributeTok{fill =}\NormalTok{ candidate), }
           \AttributeTok{stat=}\StringTok{\textquotesingle{}identity\textquotesingle{}}\NormalTok{)}
\end{Highlighting}
\end{Shaded}

By default, the groups are stacked. This can be controlled with the
position parameter, which can be \texttt{dodge} (for grouped bars) or
\texttt{fill} (stacking to 100\%):

\begin{Shaded}
\begin{Highlighting}[]
\FunctionTok{ggplot}\NormalTok{(gop2) }\SpecialCharTok{+} 
  \FunctionTok{geom\_bar}\NormalTok{(}\FunctionTok{aes}\NormalTok{(}\AttributeTok{x=}\NormalTok{state, }\AttributeTok{y=}\NormalTok{votes, }\AttributeTok{fill=}\NormalTok{candidate), }
           \AttributeTok{stat=}\StringTok{\textquotesingle{}identity\textquotesingle{}}\NormalTok{, }
           \AttributeTok{position=}\StringTok{\textquotesingle{}dodge\textquotesingle{}}\NormalTok{)}
\FunctionTok{ggplot}\NormalTok{(gop2) }\SpecialCharTok{+} 
  \FunctionTok{geom\_bar}\NormalTok{(}\FunctionTok{aes}\NormalTok{(}\AttributeTok{x=}\NormalTok{state, }\AttributeTok{y=}\NormalTok{votes, }\AttributeTok{fill=}\NormalTok{candidate), }
           \AttributeTok{stat=}\StringTok{\textquotesingle{}identity\textquotesingle{}}\NormalTok{, }
           \AttributeTok{position=}\StringTok{\textquotesingle{}fill\textquotesingle{}}\NormalTok{)}
\end{Highlighting}
\end{Shaded}

\hypertarget{line-plots}{%
\section{Line plots}\label{line-plots}}

Finally, another frequent graph is the line graph. For example, we can
plot the ascendancy of Donald Trump by looking at his vote share over
time. First, we combine the results per state with the primary schedule:
(see the tutorial on \href{R-tidy-13a-joining.md}{Joining data})

\begin{Shaded}
\begin{Highlighting}[]
\CommentTok{\# dataset 1}
\NormalTok{schedule  }\OtherTok{\textless{}{-}} \FunctionTok{read\_csv}\NormalTok{(}\FunctionTok{paste}\NormalTok{(csv\_folder\_url, }\StringTok{"primary\_schedule.csv"}\NormalTok{, }\AttributeTok{sep=}\StringTok{"/"}\NormalTok{)) }\SpecialCharTok{\%\textgreater{}\%} 
  \FunctionTok{mutate}\NormalTok{(}\AttributeTok{date =} \FunctionTok{as.Date}\NormalTok{(date, }\AttributeTok{format=}\StringTok{"\%m/\%d/\%y"}\NormalTok{))}
\NormalTok{schedule}

\CommentTok{\# dataset 2}
\NormalTok{trump }\OtherTok{\textless{}{-}}\NormalTok{ results\_state }\SpecialCharTok{\%\textgreater{}\%} 
  \FunctionTok{group\_by}\NormalTok{(state, party) }\SpecialCharTok{\%\textgreater{}\%} 
  \FunctionTok{mutate}\NormalTok{(}\AttributeTok{vote\_prop=}\NormalTok{votes}\SpecialCharTok{/}\FunctionTok{sum}\NormalTok{(votes)) }\SpecialCharTok{\%\textgreater{}\%} 
  \FunctionTok{filter}\NormalTok{(candidate}\SpecialCharTok{==}\StringTok{"Donald Trump"}\NormalTok{)}
\NormalTok{trump}

\CommentTok{\# join the two data sets (more next sessions)}
\NormalTok{trump }\OtherTok{\textless{}{-}} \FunctionTok{left\_join}\NormalTok{(trump, schedule) }\SpecialCharTok{\%\textgreater{}\%} 
  \FunctionTok{group\_by}\NormalTok{(date) }\SpecialCharTok{\%\textgreater{}\%} 
  \FunctionTok{summarize}\NormalTok{(}\AttributeTok{vote\_prop =} \FunctionTok{mean}\NormalTok{(vote\_prop))}
\NormalTok{trump}
\end{Highlighting}
\end{Shaded}

Take a minute to inspect the code above, and try to understand what each
line does! The best way to do this is to inspect the output of each
line, and trace back how that output is computed based on the input
data.

\begin{Shaded}
\begin{Highlighting}[]
\FunctionTok{ggplot}\NormalTok{(trump) }\SpecialCharTok{+} 
  \FunctionTok{geom\_line}\NormalTok{(}\FunctionTok{aes}\NormalTok{(}\AttributeTok{x =}\NormalTok{ date, }\AttributeTok{y =}\NormalTok{ vote\_prop))}
\end{Highlighting}
\end{Shaded}

\hypertarget{multiple-faceted-plots}{%
\section{Multiple `faceted' plots}\label{multiple-faceted-plots}}

Just to show off some of the possibilities of ggplot, let's make a plot
of all republican primary outcomes on Super Tuesday (March 1st):

\begin{Shaded}
\begin{Highlighting}[]
\NormalTok{super }\OtherTok{\textless{}{-}}\NormalTok{ results\_state }\SpecialCharTok{\%\textgreater{}\%} 
  \FunctionTok{left\_join}\NormalTok{(schedule) }\SpecialCharTok{\%\textgreater{}\%} 
  \FunctionTok{filter}\NormalTok{(party }\SpecialCharTok{==} \StringTok{"Republican"} \SpecialCharTok{\&}\NormalTok{ date }\SpecialCharTok{==} \StringTok{"2016{-}03{-}01"}\NormalTok{) }\SpecialCharTok{\%\textgreater{}\%} 
  \FunctionTok{group\_by}\NormalTok{(state) }\SpecialCharTok{\%\textgreater{}\%} 
  \FunctionTok{mutate}\NormalTok{(}\AttributeTok{vote\_prop =}\NormalTok{ votes}\SpecialCharTok{/}\FunctionTok{sum}\NormalTok{(votes))}

\FunctionTok{ggplot}\NormalTok{(super) }\SpecialCharTok{+} 
  \FunctionTok{geom\_bar}\NormalTok{(}\FunctionTok{aes}\NormalTok{(}\AttributeTok{x =}\NormalTok{ candidate, }\AttributeTok{y =}\NormalTok{ vote\_prop), }
           \AttributeTok{stat =} \StringTok{\textquotesingle{}identity\textquotesingle{}}\NormalTok{) }\SpecialCharTok{+} 
  \FunctionTok{facet\_wrap}\NormalTok{(}\SpecialCharTok{\textasciitilde{}}\NormalTok{state, }\AttributeTok{nrow =} \DecValTok{3}\NormalTok{) }\SpecialCharTok{+} 
  \FunctionTok{coord\_flip}\NormalTok{()}
\end{Highlighting}
\end{Shaded}

Note \textsubscript{facet\_wrap} wraps around a single facet. You can
also use \textasciitilde facet\_grid() to specify separate variables for
rows and columns.

\textbf{Exercise 2:} How could we change the plot to color the bars
according to the candidates?

\begin{Shaded}
\begin{Highlighting}[]
\CommentTok{\# Solution here}
\end{Highlighting}
\end{Shaded}

\hypertarget{themes}{%
\section{Themes}\label{themes}}

Customization of things like background colour, grid colour etc. is
handled by themes. \texttt{ggplot} has two built-in themes:
\texttt{theme\_grey} (default) and \texttt{theme\_bw} (for a more
minimal theme with white background). The package ggthemes has some more
themes, including an `economist' theme (based on the newspaper). To use
a theme, simply add it to the plot:

\begin{Shaded}
\begin{Highlighting}[]
\FunctionTok{library}\NormalTok{(ggthemes)}
\FunctionTok{ggplot}\NormalTok{(trump) }\SpecialCharTok{+} 
  \FunctionTok{geom\_line}\NormalTok{(}\FunctionTok{aes}\NormalTok{(}\AttributeTok{x =}\NormalTok{ date, }\AttributeTok{y =}\NormalTok{ vote\_prop)) }\SpecialCharTok{+} 
  \FunctionTok{theme\_economist}\NormalTok{() }\SpecialCharTok{+}
  \FunctionTok{labs}\NormalTok{(}\AttributeTok{x =} \StringTok{""}\NormalTok{, }\AttributeTok{y =} \StringTok{"Percentage that voted for Trump"}\NormalTok{)}
\end{Highlighting}
\end{Shaded}

Some links for learning more about themes:

\begin{itemize}
\tightlist
\item
  \url{https://ggplot2.tidyverse.org/reference/theme.html}
\item
  \url{https://www.datanovia.com/en/blog/ggplot-themes-gallery}
\item
  \url{http://rstudio-pubs-static.s3.amazonaws.com/284329_c7e660636fec4a42a09eed968dc47f32.html}
\end{itemize}

\end{document}
